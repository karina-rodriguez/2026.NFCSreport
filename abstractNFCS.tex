\documentclass[12pt,A4]{article}
\usepackage{authblk}
\usepackage[backend=biber,
    maxbibnames=99,
    style=authoryear-comp, % or authoryear-comp
    sortcites=true,
    sorting=nyt,
    hyperref=true,
    backref=false]{biblatex} 
\addbibresource{biblio.bib}

\title{Arts and Humanities Approaches and Requirements for Digital Research Infrastructures: Initial Findings}
\date{\normalsize{January 2026}}
\author[1]{Karina Rodriguez Echavarria}
\author[1]{Myrsini Samaroudi}
\author[2]{Arianna Ciula}
\author[2]{Neil Jakemann}
\author[2]{Matt Penn}
\affil[1]{University of Brighton}
\affil[2]{Kings Digital Lab, Kings College London}
\begin{document}
\maketitle
 
This paper presents work in progress to establish the requirements of the Arts and Humanities (A\&H) communities for a federated Digital Research Infrastructure. Within these communities, researchers are increasingly using data science and digital methods in their research and practice, as well as producing data, software and workflow outputs. The research conducted to establish the requirements of the A\&H
communities for federation involved a thorough review of established research and evidence published in relevant literature (\cite{beavan_2025a_15083396,beavan_2025b_15083632,bell_2025_17099176,Black19102024,connected,collier25,de_roure_2022_4716148,evans23,hawkins_2024_14006955,shoaib_sufi_2023_7686348,ross_2024_13710266,romanova_2021_5105938,Rodriguez22,wrro203045,Zwiggelaar22,taylor_2022_10518740,sichani_2023_8177926}). These reports where compiled and analysed thematically highlighting
insights from the A\&H subdisciplines, as well as cross-disciplinary research,
in terms of data practices, tools and workflows, collaboration and professional
development. Pilot and scoping aspects of the Infrastructure for Digital Arts and Humanities (iDAH) were the focus of the investigation.  The findings identify specific requirements for an infrastructure which addresses the needs of communities in three key areas: 1) Research Outputs, Datasets and Software; 2) Research Methods, Workflows and Tools; and 3) Culture and People. 
While acknowledging that needs vary considerably across subdisciplines and institutional access to DRI, the report identifies requirements for access, workflows/tools, data management, digital methods, collaboration, sustainability, community support, while ensuring DRI serves the members of the community to innovate also supporting them in their professional careers. The research is available as a working version (\cite{rodriguez_echavarria_2026_18350953}) and will continue to developed in the following months.
\printbibliography

\end{document}