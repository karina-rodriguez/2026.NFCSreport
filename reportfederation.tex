\documentclass[11pt]{scrartcl}
\usepackage[a4paper, top=20mm, bottom=20mm, left=20mm, right=20mm]{geometry}
\usepackage{blindtext}
\usepackage{authblk}
\renewcommand\Authfont{\normalsize}
\renewcommand\Affilfont{\normalsize}
\usepackage[helvratio=1.0]{newtxtext} % to mimic Word's scaling
\renewcommand{\baselinestretch}{1.15}  % to mimic Word's line spacing
\usepackage{microtype} % to save whatever aesthetics can be saved after mimicing Word 
\frenchspacing

\usepackage[dvipsnames]{xcolor}
\usepackage[colorlinks=true,linkcolor=blue,anchorcolor=blue,citecolor=blue,filecolor=blue,menucolor=blue,runcolor=blue,urlcolor=Mulberry]{hyperref}
\usepackage[backend=biber,
    maxbibnames=99,
    style=authoryear, % or authoryear-comp
    sortcites=true,
    sorting=nyt,
    hyperref=true,
    backref=true]{biblatex} 
\addbibresource{biblio.bib}

\newcommand\mkbibcolor[2]{\textcolor{#1}{\hypersetup{citecolor=#1}#2}}
\DeclareCiteCommand{\cite}[\mkbibcolor{blue}]
  {\usebibmacro{prenote}}%
  {\usebibmacro{citeindex}%
  [\usebibmacro{cite}]}
  {\multicitedelim}
  {\usebibmacro{postnote}}
\renewcommand*{\bibfont}{\normalfont\normalsize\sffamily}
\usepackage{enumitem}
\setenumerate[1]{label=\thesubsection.\arabic*.}
\setenumerate[2]{label*=\arabic*.}

%for the header
\usepackage{fancyhdr}
\usepackage{lastpage}
\pagestyle{fancy}
\fancyhf{}
\fancyhead[R]{Federated Digital Research Infrastructures (DRI) for the Arts and Humanities Communities Report}
\rfoot{Page \thepage \hspace{1pt} of \pageref{LastPage}}

\title{Federated Digital Research Infrastructures (DRI) for the Arts and Humanities Communities}
\subtitle{\LARGE{Insights into Communities and Digital Research Practices}}
\date{\normalsize{January 2026}}
\author[1]{Karina Rodriguez Echavarria}
\author[1]{Myrsini Samaroudi}
\author[2]{Arianna Ciula}
\author[2]{Neil Jakemann}
\author[2]{Matt Penn}
\affil[1]{University of Brighton}
\affil[2]{Kings Digital Lab, Kings College London}

\begin{document}
\sffamily
\maketitle
\section{Introduction} 
This report aims to reflect the collective voice of the Arts \& Humanities (A\&H) research communities advocating for access to Digital Research
 Infrastructure (DRI) in ways that support their digital research workflows and data management practices. 
 The (A\&H) communities bring together researchers across three
 thematic areas broadly categorised by the A\&H Research Council into: 1)
 Histories, Cultures and Heritage, 2) Creative and Performing Arts, as well as
 3) Languages and Literature. The research in these areas is
 funded mainly by the UKRI Arts and Humanities Research Council (AHRC) and
 supported by the Infrastructure for Digital Arts and Humanities (iDAH).   


iDAH gives researchers dedicated data services which cater for different disciplines
and/or types of data, including the Distributed System of Scientific
Collections (DiSSCo); the Archaeology Data Service (ADS); the Literary and
Linguistic Data Service (LLDS); the British Library Research Repository; the
Museum Data Service (MDS), the Heritage Science Data Service (HSDS), and the
Enact Practice Research Data Service. In addition, iDAH engages with the A\&H
research and practice communities to improve skills and promote the overall
culture around data, tools and software, for instance, in terms of practices,
value and role in decision making, through initiatives such as the \href{https://www.diskah.org/}{Digital
Skills in Arts and Humanities (DISKAH) Network}, the \href{https://www.ccpahc.ac.uk/}{Collaborative Computational Project for Arts, Humanities, and Culture (CCP-AHC)} and \href{https://www.turing.ac.uk/research/research-projects/dataculture-building-sustainable-communities-around-arts-and-humanities}{DataCulture}.  

The research conducted to establish the requirements of the A\&H
communities for federation involved a thorough review of established research and evidence published in relevant literature (\cite{beavan_2025a_15083396,beavan_2025b_15083632,bell_2025_17099176,Black19102024,connected,collier25,de_roure_2022_4716148,evans23,hawkins_2024_14006955,shoaib_sufi_2023_7686348,ross_2024_13710266,romanova_2021_5105938,Rodriguez22,wrro203045,Zwiggelaar22,taylor_2022_10518740,sichani_2023_8177926}). These reports where compiled and analysed thematically highlighting
insights from the A\&H subdisciplines, as well as cross-disciplinary research,
in terms of data practices, tools and workflows, collaboration and professional
development. 
% Thereafter, these insights informed a gathering exercise on
% the requirements of data service providers regarding federation, based on their
% existing provision, patterns of data use, and future aspirations and priorities
% for federation. Both of these research components are described in the
% report. 

It should also be emphasised that the Arts \& Humanities (A\&H) research communities
are not a single, homogeneous group. Research needs vary considerably across
subdisciplines and are further shaped by career stage, access to institutional
infrastructure, collaborative contexts, and levels of digital skills.
Acknowledging and addressing diverse requirements is therefore critical to
ensure that a federated DRI is accessible, inclusive and equitable, and that it
can effectively support the broad A\&H research and practice communities.  

\section{Requirements from Literature Survey}
\subsection{Research Outputs, Datasets and Software}  
\textbf{Communities use and produce heterogeneous datasets of varying complexities involving multiple formats, parts, sizes and data types. While textual data has the strongest representation, a large percentage of researchers work with multidimensional data, including born digital data (e.g. spatial, numerical, visual, audiovisual) and sensor data (real-time). The resulting datasets are often well curated in accordance with the FAIR (findable, accessible, interoperable, reusable) principles, enabling the structuring and connecting of knowledge. Researchers also generate a range of software artefacts, particularly interactive applications, mobile apps, and server-side applications. Researchers and practitioners are interested in the use of DRI for the retention, use/reuse and preservation of these outputs.}

\begin{enumerate}[resume]
\item DRI workflows should accommodate heterogeneous datasets with diverse data types, varied and evolving sizes, ensuring that processing pipelines, tools, and interfaces effectively support (\cite{Rodriguez22,de_roure_2022_4716148}; \cite[4]{bell_2025_17099176}; \cite[38]{taylor_2022_10518740}). 
File formats and sizes for workflows, such as migration, movement and emulation, should be recommended to the community 
(\cite[5]{wrro203045}).


\item DRI should be capable of sustainably hosting and preserving complex, full‑stack scholarly outputs (e.g., data, software, interfaces, documentation) (\cite[14]{de_roure_2022_4716148}).   

\item DRI should embed copyright policy into technical workflows so that systems can automatically determine whether an asset can be shared, reused, or distributed. Intellectual Property Rights must be agreed via a governance mechanism, for instance, centrally and by data provider (\cite{de_roure_2022_4716148}).


\item DRI should support licensing for granular release of content, enabling researchers or data owners to specify conditions for access, downloading, and reuse on a per‑request basis (\cite[5]{wrro203045}).

\item DRI should support the adoption of a robust Data Management Plan (DMP) following best practices and clear data policies 
(\cite[50]{taylor_2022_10518740}; \cite[7,9]{hawkins_2024_14006955}).
Researchers require guidance on how software should be managed, documented, stored, maintained, and easily made accessible and findable, including beyond the funding period 
(\cite[40,41]{romanova_2021_5105938}; \cite[40,41]{taylor_2022_10518740}; \cite[63]{shoaib_sufi_2023_7686348}; \cite[1,29]{hawkins_2024_14006955}).   
\item DRI should support moving data via batch file transfers or other automated mechanisms. Long-lived or refreshable access tokens can support orchestration across systems 
(\cite[5]{collier25}).  

\item DRI should support a minimum level of centralisation and integration for storing, analysing, and sharing data, while exploring the potential of decentralisation technologies (\cite[1,12]{hawkins_2024_14006955}).  
\end{enumerate} 

\noindent\textbf{Communities prioritise sharing and enabling access to both data and software outputs. Once data are openly available on web platforms, interoperability is viewed as a key mechanism for connecting datasets across institutions and disciplinary domains.}

\begin{enumerate}[resume]
\item DRI should support researchers involved in funded research in depositing their data and software outputs in appropriate repositories upon project completion (\cite[92]{taylor_2022_10518740}). These outputs should have clear DOIs and clear licensing and support career development needs, e.g. REF
(\cite[38]{taylor_2022_10518740}; \cite[7]{hawkins_2024_14006955}).

\item DRI should support discoverability, access, use/reuse, and interactivity with datasets, including access to data via APIs, direct download, SPARQL, search and visualisation interfaces, and optimisation for search engines, so that research is indexed, visible, and discoverable 
(\cite[12,18]{de_roure_2022_4716148}; \cite{Rodriguez22}; \cite[5]{wrro203045}).  

\item DRI should support technologies for linking and supporting discoverability by aggregators. Data should be linked across the sector nationally and internationally to unlock research potential (\cite[11]{de_roure_2022_4716148}; \cite{Rodriguez22}; \cite[9,11]{Black19102024}; \cite[30]{taylor_2022_10518740}; \cite[5,20]{Zwiggelaar22}; \cite[14]{hawkins_2024_14006955}).   

\item DRI should establish minimum requirements for data integrity and standardisation, and agree on ontologies, taxonomies, and vocabularies for indexing, naming, and cataloguing. Workflows with Machine learning methods should be explored (\cite[53]{de_roure_2022_4716148}; \cite[14]{hawkins_2024_14006955}). 
\end{enumerate}

\noindent\textbf{Communities have as a key consideration the long-term data preservation of the data.}

\begin{enumerate}[resume]
\item DRI needs to support the long-term preservation of data and software outputs through national and institutional services (\cite[8]{
beavan_2025b_15083632}).

\item DRI should offer support at the start of the project regarding standards and guidelines for archiving and preservation that facilitate future data science endeavours, improving accessibility and reducing data wrangling effort; as well as at the end of the project when preparing derived datasets for dissemination 
(\cite[4,18]{de_roure_2022_4716148}).

\item DRI should consider adopting models for digital preservation readiness, such as the NDSA Levels of Digital Preservation or the Digital Preservation Coalition’s Rapid Assessment Model (DPC RAM) (\cite[4,21]{Zwiggelaar22}).  
\end{enumerate}

\noindent\textbf{Communities have a major concern for the security of the data, given the sensitive nature or other aspects related to curation and use of these datasets.} 

\begin{enumerate}[resume]
\item DRI should address security risks and be capable of storing, managing, and providing access to sensitive data 
(\cite{de_roure_2022_4716148}; \cite[5]{wrro203045})
(De Roure et al., 2022; Jackson et al., 2023, p. 5). Trusted environments should be available for sensitive data, including historically and commercially sensitive data, as well as stray data. These TREs should have policies tailored to sensitive content rather than a focus on personal data to minimise excessive barriers to research 
(\cite[4]{de_roure_2022_4716148}; \cite[5]{Zwiggelaar22}).

\item DRI should be compliant with cybersecurity and governance requirements (
(\cite[5,6]{collier25}).  
\item DRI should reduce the user friction introduced by DRI Cybersecurity approaches, such as Multi-Factor Authentication, by making MFA checks more innovative and more targeted to particular situations 
(\cite[21]{collier25}).  
\end{enumerate}

\subsection{Research Methods, Workflows and Tools}  

\noindent\textbf{Communities can benefit from reproducible workflows for data science approaches.}

\begin{enumerate}
\item DRI should support scalable and reproducible data science and AI research workflows, including efficient ways to sort, link, understand, and visualise heterogeneous data in a systematic, configurable, and reproducible manner (\cite[12]{sichani_2023_8177926}). In turn, these workflows could significantly accelerate research and promote cumulative data enrichment 
(\cite[4]{de_roure_2022_4716148}; \cite[43]{shoaib_sufi_2023_7686348}).  

\item DRI should provide supported tools and environments for research workflows and pipelines of connected processes and algorithms using virtualisation tools such as virtual machines, Docker, Singularity containers and Kubernetes virtual clusters 
(\cite[4]{de_roure_2022_4716148}). 
Workflows are likely to migrate between physical infrastructures 
(\cite[14]{collier25}).

\item DRI should avoid supplier/vendor lock-in, as it affects hardware and software and requires time and additional cost to rewrite code and migrate (\cite[3,4]{bell_2025_17099176}).  
\end{enumerate}

\noindent\textbf{Communities increasingly conduct large-scale analyses that demand computational resources and scalable approaches. }

\begin{enumerate}[resume]
\item DRI should provide easier access to dedicated High-Performance systems, scalable cloud services, and processing power to handle large-scale data
(\cite[62]{de_roure_2022_4716148}; \cite[2]{bell_2025_17099176}; 
\cite[72]{shoaib_sufi_2023_7686348}).   

\item DRI should provide more endpoints and opportunities to generate analysis-ready data for Machine Learning and to run subsequent analyses at the data provider (e.g., non-consumptive research/analytics) 
(\cite[4,66]{de_roure_2022_4716148}; \cite[5]{wrro203045}).

\item DRI needs for accounting approaches to track CPU usage, project allocations, detailed GPU storage usage, user-facing energy usage, carbon tracking data and more 
(\cite[14,15]{collier25}).  

\item DRI needs to mitigate AI-related risks via an ethics board, an ethical governance framework, a human framework for testing and training of algorithms (human-in-the-loop) and a training programme to address concerns around the trust and reliability of AI, including its accuracy and data quality, unexplained black-box operations, bias and privacy (\cite[4]{Zwiggelaar22}).   
\end{enumerate}

\noindent\textbf{Communities predominantly rely on graphical and visually rich interfaces for workflows and interfaces, with limited familiarity with command-line tools and programming languages.}

\begin{enumerate}[resume]
\item DRI should provide an intuitive and visual interface to enable data-intensive research 
(\cite[4]{wrro203045}), 
including tailored interfaces for editing, analysis, visualisation/streaming and interaction with data, commentary, citation, as well as extraction (\cite{Rodriguez22}; \cite[44]{de_roure_2022_4716148}; \cite[5]{wrro203045}). 
Accessibility standards should be considered 
(\cite[66]{de_roure_2022_4716148}; \cite[5]{wrro203045}).  

\item DRI should enable support for interactive/urgent computing interfaces (notebook-style), which are familiar to AHC software developers (\cite[4]{bell_2025_17099176}). 
\end{enumerate}

\noindent\textbf{Communities favour open science and open data practices.}

\begin{enumerate}[resume]
\item DRI should support open data and open-source cultures and more established ways for sharing access to high-quality data with appropriate licensing, interoperability standards and documentation 
(\cite[17]{de_roure_2022_4716148}; \cite[2]{bell_2025_17099176}; \cite[58]{shoaib_sufi_2023_7686348}; \cite[1,2]{hawkins_2024_14006955}).   

\item DRI should encourage FAIR data principles and Creative Commons open licences with options for a restricted licence if data would otherwise not be ingested (\cite[4,21]{Zwiggelaar22}).  

\item DRI should promote open research principles tailored to arts and humanities research approaches 
(\cite[51]{taylor_2022_10518740}).
  
\end{enumerate}



\noindent\textbf{Communities want user journeys and use cases for digital methods as a means to support guided access of DRI.}

\begin{enumerate}[resume]
\item DRI should adopt taxonomies such as the TADRIAH taxonomy to categorise popular tasks with the view of guiding researchers. DRI should advise on appropriate computing architectures and frameworks for their deployment, according to the established requirements of these systems (e.g., data access, CPU/GPU-intensive).   

\item DRI should elicit and showcase usable pathfinder projects and demonstrator/pilot applications for popular tasks to support uptake by the community, learning by other people’s workflows and overall experiences (\cite[3,9]{bell_2025_17099176}; \cite[5]{evans23}).

\item DRI should provide a brokerage process or switchboard at the point of access to help users identify the most appropriate resource/federated access and guide users to various components of the infrastructure according to their needs 
(\cite[5]{bell_2025_17099176}; \cite{connected}).  
The process should include matching datasets with a suitable repository and guiding users through various computing techniques for data cleaning, text/data mining, data visualisation, immersive technologies, database design, statistics, machine learning, geospatial data, NLP, topic modelling, computer vision, and probabilistic thinking 
(\cite{connected}; \cite[38]{shoaib_sufi_2023_7686348}).   

\item DRI should consider how Artificial Intelligence (AI) methods and Large Language Models (LLMs) can assist researchers in their tasks 
(\cite[2]{bell_2025_17099176}).


\item DRI should support different levels of use, covering users' varied needs, paired with a clear funding and access model for infrastructure 
(\cite{Rodriguez22}; \cite[13]{Black19102024})
\end{enumerate}

\noindent\textbf{Communities employ a wide range of research practices and methodologies, including ‘living’ practices, to produce outputs and datasets. Hence, they carry out a wide range of processes on datasets throughout the research lifecycle. }

\begin{enumerate}[resume]
\item DRI should support a more structured and granular approach to versioning, considering both work (which is developmental) and final outputs, and reflecting on the relationship between changes in the data and particular actions or methodologies 
(\cite[53]{de_roure_2022_4716148}; \cite[4]{wrro203045}). 
The UK Registry for the Research Activity Identifier (RAiD) and DataCite should be considered, as they already support versioning of DOIs for multiple versions of datasets (\cite[5]{evans23}).  
\end{enumerate}

\subsection{Cultures and People}

\noindent\textbf{Communities collaborate across institutional boundaries and with non-academic communities at both national and international levels.}  

\begin{enumerate}
\item DRI should allow for authorship of research outputs by individuals and teams across institutions (\cite[5]{evans23}). Metadata based on existing taxonomies, such as the Contributor Roles Taxonomy (CRediT), should be considered.  

\item DRI should support access by teams formed of national and international academic and non-academic users, including access to functionalities such as data transfer/migration, permission management, and usage monitoring. Copyright and licensing of DRI-related resources should support agreements among multiple institutions 
(\cite{de_roure_2022_4716148}; \cite[5]{Zwiggelaar22}). Collaboration can be enabled via ‘short-lived collaborations' and ‘virtual organisations’ that bring together staff from multiple institutions, often in different countries 
(\cite[7]{collier25}).

\item DRI should consider reusable tools and approaches to share fine-grained information with services about a researcher, beyond eduPersonTargetedID tied to a name, email address 
(\cite[6]{collier25}).


\item DRI should encourage cross-disciplinary national and international collaborations through community engagement activities, such as hackathons 
(\cite[4]{de_roure_2022_4716148}; \cite[2]{bell_2025_17099176}; \cite[15]{romanova_2021_5105938}; \cite{sichani_2023_8177926}). 
These activities should promote collaboration among various professional roles, including DRTPs, RSEs, data scientists, and researchers 
(\cite[8]{de_roure_2022_4716148},\cite{beavan_2025a_15083396}; \cite[30]{taylor_2022_10518740}), as well as with other sectors such as the creative industries 
(\cite{romanova_2021_5105938}).  
\end{enumerate}

\noindent\textbf{Communities benefit from active connection, nurturing, as well as inclusive and sustained growth.}  

\begin{enumerate}[resume]
\item DRI should develop and sustain inclusive communities of practice beyond disciplinary domains, such as History, Language, Digital Humanities and Archaeology and academic audiences (\cite[10,26]{taylor_2022_10518740}).   

\item DRI should be supported by networking, approaches for collaborating, skills, formal and informal training, internships, and placements to support people at various career stages and in different organisations 
(\cite[17,72]{de_roure_2022_4716148}; \cite[2]{bell_2025_17099176}; \cite[20,21]{romanova_2021_5105938}; \cite[34,35]{shoaib_sufi_2023_7686348}; \cite[47]{taylor_2022_10518740}).   

\item DRI should invest long-term and sustainably in capacity, training and bridging digital literacy divides, especially boosting skills in digital methods and technologies 
(\cite[4,11]{de_roure_2022_4716148}; \cite[10]{Black19102024}; \cite[22]{collier25}; \cite[6]{evans23}; \cite[73]{shoaib_sufi_2023_7686348}; \cite[12,27]{sichani_2023_8177926}; \cite[7]{hawkins_2024_14006955}). 
Environmental awareness for doing digital research should also be considered 
(\cite[28]{sichani_2023_8177926}; \cite[6]{evans23}).  
\end{enumerate}

\noindent\textbf{Communities consider Digital Research Technical Professionals (DRTPs) to be key stakeholders to expand access to DRI.}  

\begin{enumerate}[resume]
\item DRI should ensure continuous equitable collaboration with dRTPs, including dedicated and one-to-one ‘clinic’ style support 
(\cite[48]{de_roure_2022_4716148}; \cite[8]{bell_2025_17099176}; \cite[10]{Black19102024}; \cite[17]{romanova_2021_5105938}; \cite[12]{sichani_2023_8177926}; \cite[33]{taylor_2022_10518740}; \cite[1,2]{hawkins_2024_14006955}).

\item DRI should leverage community managers to broker access to DRTPs/RSEs and widen engagement with DRI 
(\cite[10]{bell_2025_17099176}).

\item DRI should ensure broader availability of dRTP roles 
(\cite[2]{bell_2025_17099176}; \cite[48,73]{shoaib_sufi_2023_7686348}; \cite[10]{sichani_2023_8177926}).  
\end{enumerate}


\noindent\textbf{Communities see professional development across roles as essential to realising the full potential of DRI.}

\begin{enumerate}[resume]
\item DRI should invest in the human aspect to make infrastructure viable, including widening opportunities for individuals with clear progression paths, funding beyond projects and improved salaries, as well as career development opportunities 
(\cite[71]{de_roure_2022_4716148}; \cite[13,14]{romanova_2021_5105938}; \cite[8]{sichani_2023_8177926}; \cite[33]{taylor_2022_10518740}).   
\end{enumerate}

\noindent\textbf{Communities consider the evaluation frameworks, particularly the Research Excellence Framework (REF), as crucial in shaping researchers' careers. } 


\begin{enumerate}[resume]
\item DRI should support submissions to national evaluation frameworks, including framing data and software as outputs and allowing for the illustration of the research journey 
(\cite[4]{wrro203045}; \cite[71]{shoaib_sufi_2023_7686348}; \cite[41]{taylor_2022_10518740}). 

\item DRI should provide portable access to its infrastructure and allow researchers’ outputs to be portable. Policies should be flexible to accommodate staff mobility (\cite[37]{taylor_2022_10518740}) as researchers transition between academia, independent research organisations (IROs), and other sectors, often early in their careers. 

\item DRI should allow for software produced during research to be accessed as a research output, with software citation and reproducibility underpinning impact metrics 
(\cite[7]{sichani_2023_8177926}).  

\end{enumerate}

\noindent\textbf{Sustainability, in its various forms, is a central priority for the community.}   

\begin{enumerate}[resume]
\item DRI should define sustainability in terms of environmental, labour, and collaboration practices 
(\cite[1]{ross_2024_13710266}).   

\item DRI should monitor the environmental impact of infrastructure with a long-term aim of carbon neutrality, where achievable and considering the risks of exacerbating unsustainable resource consumption 
(\cite[7]{bell_2025_17099176}; \cite[13]{Black19102024}; \cite[29]{ross_2024_13710266}; \cite[21]{Zwiggelaar22}; \cite[2]{hawkins_2024_14006955}).  

\item DRI should consider different funding scenarios and find a balance, given that larger organisations usually attract bigger funding 
(\cite[21]{romanova_2021_5105938}). 
DRI should consider supporting users post-project completion
(\cite[14]{de_roure_2022_4716148}; \cite[41]{taylor_2022_10518740}). 
 

\item DRI should reflect upon the success and failure of previous infrastructure to successfully develop future strategies 
(\cite[21]{romanova_2021_5105938}).

\end{enumerate}
\printbibliography
\end{document}
